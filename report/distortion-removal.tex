% vim:ft=tex
% rubber: module xelatex
\subsection{Distortion removal}

Initial investigation:\\
- "Straight lines ought to be straight" - \cite{straightlines} .\\
- "Applying and removing lens distortion in post production" - \cite{postproduction} .\\
- The algorithm (and library implementing it) which we shall refer to as IPOLdistortion is based on \cite{algebraic-distortion} .\\\\

Notes from \cite{algebraic-distortion}
- Estimation of the lens distortion parameters is based on the rectification of lines in the image.\\
- The lens distortion parameters are obtained by minimizing a 4 total-degree polynomial in several variables.\\
- "in this paper we will always take the center of the image as distortion center"\\
- "Obviously, the global minimum of the new functional corresponds to the trivial solution k = 0. To avoid this problem, usually the first distortion parameter is set to one."\\
- The proposed energy function = a distortion error measure function, which is a real polynomial in the variable k. Minimizing a polynomial in several variables can be reduced to compute the solutions of an algebraic system of equations, namely the one generated by its gradient.\\
- "The distortion parameters k are computed setting k0 = 1. In order to yield undistorted points as close as possible to the distorted ones we estimate a zoom factor to minimize the sum of the square distance between the distorted and the corrected (undistorted) points."\\
- Working of the basic algorithm:\\
1. The algorithm initializes k = (1,0,...,0)T.\\
2. Choose any pair p , q ∈ Z ( 1 ≤ p, q ≤ Nk ) and we optimize kp, kq using the proposed algebraic technique.\\
3. Update k using a zoom factor such that distorted and undistorted points are as close as possible.\\\\

Our implementation and extension of the algorithm:\\
1. We use opencv to automatically extract corners from a distorted chessboard image.\\
2. We reconstruct segments that should be straight lines from these points, using a naive algorithm which relies on the fact that opencv finds corners in a set order; the algorithm only works if all the internal points are found.\\
3. For optimal distortion removal, we use use multiple lines, both vertical and horizontal.\\
4. The actual distortion removal consists of first modelling the distortion (by determining the radial distortion coefficients) and then correcting the image based on that model. This is done using an open-source ANSI C library which uses an optimizer to determine the undistorted coefficients by minimizing the error between the radial distortion model and the image data.\\
5. Note that this usage of the algorithm only works within certain constraints: only on relatively 'clean' images of [distorted] chessboards (due to opencv), and only on uncompressed .tif images (due to the IPOLdistort library).\\
5.a) (We our work to be an extension of the functionality of IPOLdistortion, which requires the user to input line segments by hand. Further work would obviously extend the functionality to cover non-chessboard images.)\\
6. SEE the IPOLdistortion readme for a good (simple) explanation; it is based on \cite{algebraic-distortion}.\\\\

Other useful references\\
\begin{verbatim}
http://dl.acm.org/citation.cfm?id=1924387
http://scien.stanford.edu/pages/labsite/2007/psych221/projects/07/geometric_distortion/project.htm
http://www.embedded-vision.com/industry-analysis/technical-articles/2011/05/14/lens-distortion-correction
\end{verbatim}

Distortion removal experiments:\\
-> We created a suite of 22 different 8x8 chessboard images. These included variations on rotation, position, barrel and pincushion distortion, other forms of distortion, and noise.\\
-> We compare calculated  lens distortion parameters (kappas) on different chessboard types.\\
-> We also checked that chessboards of varying dimensionality can be used. This is successful for chessboards with side lengths at minimum 4, and up to length 16 or more (assuming the size of each chessboard square remains the same).
-> Finally, comparison with camera calibration results: we generated lists of lines for each face of three images, 0027, 0027 with barrel distortion, and 0027 with pincushion distortion. We generated straight lines from the image calibration points, which obviously ought to lie on straight lines. For each case, we used all horizontal lines and the two outer vertical lines from each face. We ran distortion removal on these, and compared the kappa-0, kappa-2 and kappa-4 output of the distortion algorithm to the kappa-1 found by calibration.\\


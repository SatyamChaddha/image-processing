% vim:ft=tex
% rubber: module xelatex

\subsection{Distortion removal}

% http://dl.acm.org/citation.cfm?id=1924387
% http://scien.stanford.edu/pages/labsite/2007/psych221/projects/07/geometric_distortion/project.htm
% http://www.embedded-vision.com/industry-analysis/technical-articles/2011/05/14/lens-distortion-correction

% "Straight lines ought to be straight" - \cite{straightlines} .\\
% "Applying and removing lens distortion in post production" - \cite{postproduction} .\\

We investigated forms of distortion removal independent of camera calibration. One particularly useful and simple open-source ANSI C library we discovered was provided by the authors of \cite{algebraic-distortion}. Herein we shall refer to this IPOL library and the algorithm it describes as IPOLdistortion. This implementation estimates the lens distortion parameters of a camera (or image) based on the rectification of lines in the images, as described in e.g. \cite{straightlines}, but with some innovations.

\cite{algebraic-distortion} find the lens distortion parameters by minimizing a 4 total-degree polynomial in several variables. The mathematical theory behind their implementation (for which see 

?????

) leads the authors to only consider kappa-0, kappa-2 and kappa-4. See the paper cited for a full explanation. The algorithm functions by setting the first distortion parameter to 1 (to avoid the trivial solution kappa-0 = 0), and then minimising a distortion error measure function in the form of an energy function of the authors' design. This function is a real-valued polynomial in the variable kappa. To minimise the function, IPOLdistortion finds the solutions to the algebraic system of equations generated by its gradient. The implementation also introduces a "zoom factor" minimising the distance between distorted and corrected points, in order to to create corrected points as close as possible to the distorted ones \cite{algebraic-distortion}.

\subsubsection{Implementation notes}

The IPOLdistort library requires that users manually select points on the straight line for the algorithm to work on. This is both an arduous process and susceptible to human error. Furthermore, \cite{algebraic-distortion} state that for the maximum efficacy, as many straight lines as possible should be used. Therefore in our implementation, we combine the distortion correcting functionality of IPOLdistort with the line extraction functionality of OpenCV. 

We use OpenCV to automatically extract detected corner points from a (distorted) chessboard image. From these corner points, and our knowledge of the chessboard dimensionality (number of squares per side), we can reconstruct segments which "should" be straight lines in the image. The algorithm we use for this is relatively naive, relying on the fact that OpenCV returns corners in a fixed order. The algorithm is likely to fail if this list is out of order for some reason, or if a subset of points is not returned. For the best performance, the algorithm should take multiple vertical and horizontal lines as input. We therefore provide it with all the horizontal lines returned by the OpenCV code, as well as two vertical lines reconstructed from these.

The actual distortion removal functionality provided by IPOLdistortion consists of first modelling the distortion and then correcting the image based on that model.



blah blah
This is done using an open-source library which uses an optimizer to determine the undistorted coefficients by minimizing the error between the radial distortion model and the image data.
5. Note that this usage of the algorithm only works within certain constraints: only on relatively 'clean' images of [distorted] chessboards (due to opencv), and only on uncompressed .tif images (due to the IPOLdistort library).\\
5.a) (We our work to be an extension of the functionality of IPOLdistortion, which requires the user to input line segments by hand. Further work would obviously extend the functionality to cover non-chessboard images.)\\
6. SEE the IPOLdistortion readme for a good (simple) explanation; it is based on \cite{algebraic-distortion}.\\\\



\subsubsection{Experimental process.}

\paragraph{Experimental parameters.}
To test our program, 

Distortion removal experiments:\\
-> We created a suite of 22 different 8x8 chessboard images. These included variations on rotation, position, barrel and pincushion distortion, other forms of distortion, and noise.\\
-> We compare calculated  lens distortion parameters (kappas) on different chessboard types.\\
-> We also checked that chessboards of varying dimensionality can be used. This is successful for chessboards with side lengths at minimum 4, and up to length 16 or more (assuming the size of each chessboard square remains the same).
-> Finally, comparison with camera calibration results: we generated lists of lines for each face of three images, 0027, 0027 with barrel distortion, and 0027 with pincushion distortion. We generated straight lines from the image calibration points, which obviously ought to lie on straight lines. For each case, we used all horizontal lines and the two outer vertical lines from each face. We ran distortion removal on these, and compared the kappa-0, kappa-2 and kappa-4 output of the distortion algorithm to the kappa-1 found by calibration.\\

\paragraph{Experimental results.}
(INSERT CHARTS HERE)

Note that \cite{algebraic-distortion} state, "we will always take the center of the image as distortion center".





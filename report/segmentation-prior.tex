% vim:ft=tex
% rubber: module xelatex
\subsection{Image segmentation}

Image segmentation is useful not just in the more abstract areas of machine vision, but important real-world domains such as medical imaging and face recognition tools. We can roughly divide segmentation algorithms into two classes: those based on local or global differences in pixel properties, and those which consider wider properties of the image.

Algorithms segmenting images based on pixel intensity or colour variations are often simpler and faster. For instance, an image can be segmented based on the histogram of its pixel properties; the peaks are assumed to correspond to discrete regions in the image. This can be done recursively to identify subregions; see for example \cite{recurse-segment}. Images can alternatively be modelled as a weighted undirected graph for the purposes of segmentation. Nodes corresponding to neighbouring pixels are joined by an edge with a weight which is higher the more dissimilar the pixel colours or intensities are. The graph can then be cut or partitioned in some way to separate the regions of the image. One example of this is described in \cite{image-graph}.

There are also many algorithms that use the broader mathematical properties of an image to perform segmentation. For instance, images can be segmented based on detected edges under the assumption that edges lie on the boundaries of image regions. \cite{edge-segment} describe `junction' features based on the spatial coincidence of variously `straight' or `curved' edges. There are also a broad variety of approaches to segmentation based on solving partial differential equations over the image. For example: curve propagation by minimising the potential of some cost function; discretised parameterisation of contours; diffusion filters; level setting; and Gaussian pyramid decomposition. Some of these are discussed in \cite{efficient-segmentation}.

For a general idea of the relative performance of modern segmentation algorithms, consider the publically-available benchmark provided by the Berkeley Database of Human Segmented Natural Images (see \cite{seg-database}), which contains of a subset of 300 of the database's segmented images. Using a training set of 200 images and a test set of 100 images, it was found that the best segmentation algorithms were Boosted Edge Learning, contour algorithms localising junctions in images, and algorithms computing global probability of boundaries. Also highly rated were methods using `ultrametric' contour maps associated to families of nested segmentations, and algorithms using combinations of local brightness and texture gradients to detect boundaries.

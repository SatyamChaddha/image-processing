% vim:ft=tex
% rubber: module xelatex
\subsection{Face features extraction}
We have implemented the SURF extractor, described in \cite{SURF},
which is an attempt to improve upon the first scale-invariant feature
extraction algorithm, SIFT, described in \cite{SIFT}. The
implementation is inspired by the OpenSURF implementation, described
in \cite{OpenSURF}, and the code structure is quite similar.
Interesting points:

\begin{itemize}
\item The response layers and filter sizes are taken from \cite{SURF}.
  This means the range of possible values for the octaves and interval
  parameters are somewhat limited (resp. 1-5 and 3-4). The OpenSURF
  implementation simply ignores the invervals parameter in the
  calculations.
\item When estimating the determinant of the Hessian matrix (from the
  filter response values), a weight of $0.9$ for the $D_{xy}$
  direction is used. This comes from the article.
\item The three implementations (ours, OpenCV and OpenSURF) use
  somewhat different threshold values. OpenCV does not area normalise
  the values (I think; the code is quite convoluted), which might be a
  good bet for why those are different. OpenSURF represents grey scale
  values as floating point values between 0 and 1, whereas we
  represent them as 0-255 int values, which should account for the
  values here.
\item The interpolation between points to get subpixel accuracy for
  interest points follows the method in \cite{inv-features}, which
  amounts to solving a linear system using pixel differences as
  approximations for derivatives. In \cite{SURF}, it is implied that
  this should be used to iteratively interpolate keypoint locations,
  and discard those that do not converge. However, both OpenCV and
  OpenSURF seem to just use the interpolation deltas to discard
  keypoints that interpolate to a different point, and no
  interpolation is done. We do the same; though for some reason, the
  interpolation seems to make the results worse (i.e. fit worse with
  the other implementations).
\end{itemize}

% vim:ft=tex
% rubber: module xelatex
\subsection{Camera calibration}

- We implemented Tsai calibration from \cite{TSAI}, choosing it as a historically proven and well-documented form of camera calibration.\\\\
- The calibration object we used takes the form of two adjacent faces of a cube, painted with circular calibration points; the left face has 35 calibration points and the right face 28 calibration points.\\\\
- The type of camera calibration which Tsai describes as 'the monoview noncoplanar case' is performed here - the calibration object has points on multiple planes and calibration is performed with a single image/view.\\\\
Implementation procedure:
\begin{itemize}
  \item For the following we take (0,0) to be the top-left pixel position of an image window or frame.
  \item We initially used ImageJ to segment an image and extract the 63 calibration point features.
  \item The calibration image data is now automatically determined from the image, using feature extraction algorithms described elsewhere in this report. Using the image size and a list of measurements of the calibration points in the world coordinate system, our program is able to automatically map the image points to the known locations (in world coordinates, in inches) of measured points on the object. This is done by a naive mapping which only works when two conditions are met: (a) No points from the left face in the image are further to the right (i.e. have higher x values than) any points from the right face. (b) The two 'outermost' calibration points on each face are those that are closest to the two image corners on that side. Together, these conditions mean that as the angle of the photograph (and thus the camera coordinate system) deviates more than 45 degrees from a "level" position in the world coordinate system, the calibration algortihm will become likely to fail.
  \item The calibration function derives the transformation matrix [R | T] to convert world coordinates into camera coordinates, and finds the scale factor sx, and the internal camera values for the focal length f and the lens distortion error [kappa]. This all follows the Tsai paper.
  \item Radial distortion is calculated as far as [kappa 1]. The literature suggests that this is sufficient for reasonably high accuracy using cameras without significant lens distortion.
  \item Note: for some time, my approximation for kappa was broken. The gradient descent was broken so that it stuck at 1 times 10 to the power -8. This is now fixed.
  \item Some difficulties were encountered in implementing the camera calibration. One problem was that in following the Tsai paper, the author states that the final calibration stage (finding f, kappa and the z-component of the translation vector) for the monoview noncoplanar case is "exactly the same as that for the coplanar case". However, it is not explained that the equation (15) for the coplanar case is derived by setting zw = 0 (due to the fact that the z position is identical for points in the plane). This does not, of course, follow for the noncoplanar case, so the equation must be re-derived from earlier equation (8b), and this was not immediately obvious.
  \item Back-projection of rays also proved a difficult problem.
\end{itemize}
Other notes\\
- Experiments... Patrice suggests we "create a line (using image points and known points of the camera geometry, optical centre, etc) which intersects with the calibration plane in a given point (e.g. Z=0 in world coordinates)."\\
- Assignment states "You need to evaluate the accuracy of the calibration data by comparing the true and reconstructed 3D coordinates of the disks in the calibration pattern. You may find useful to provide the average and standard deviation of the reconstructed 3D points errors. You may use the radius of ambiguity measure as in Tsai publication."\\
- Back-projection now works... to a degree. The error between the real 3d point and the predicted 3d point is on the order of ~1.2 inches. This is so huge that I suspect systematic error - either a problem with the physical calibration object or (much more likely) a problem in my implementation of the algorithm. Note that x and y error (the distance ALONG each face) is much more substantial than z error.\\
\\
Calibration experiments:\\
-> On eight images (2 low grade, 2 high grade, 2 low grade distorted, 2 high grade distorted)...\\
-> Compare backprojection accuracy... mean, variance, worst hit, best hit, breakdown into (x,y,z) \\
-> Compare kappa calculated (should be higher with lens distortion... but using actual fisheye distortion may be so high that the polynomial kappa model is unsuitable for it, as \cite{straightlines} (I think it is) suggests).\\
-> Subjectively, look at the resulting translation and rotation matrices and consider what they mean, as well as the sx and focalLength.\\

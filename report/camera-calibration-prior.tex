% vim:ft=tex
% rubber: module xelatex
\subsection{Camera calibration}

Camera calibration or `resectioning' - the calculation of the intrinsic and extrinsic parameters for a particular camera - has evolved since the two-stage algorithm of Tsai's seminal paper, \cite{TSAI}. \cite{ZHANG} classifies techniques as either \emph{photogrammetric calibration} (calibration performed using a calibration object with known 3D geometry) or \emph{self-calibration} (calibration performed by taking multiple images of a static scene using a camera with fixed internal parameters). Zhang's own technique is essentially a relaxed version of photogrammetric calibration, using a single planar pattern.

The direct linear transformation (DLT) method, as laid out in e.g. \cite{direct-linear}, solves the problem of camera synchronisation by error minimisation over similarity relations written in homogeneous equation form. This is useful for applications such as \emph{a posteriori} video image synchronisation. \cite{direct-linear} find the phase difference, then choose one recording to be the reference and synchronise the second using cubic spline interpolation.

Self-calibration or auto-calibration allows camera parameters to be calculated directly from multiple uncalibrated images, without reference to some known 3D object. An early discussion of this is \cite{self-calib}. Innovations to enable the self-calibration of cameras despite varying or unknown internal camera parameters have also been proposed. For example, \cite{self-calibration-unknown} demonstrates a method for retrieving a reconstruction from image sequences obtained with uncalibrated zooming/focusing cameras. It is possible to automatically find the minimum sequence length for self-calibration (depending on the set of constraints) to detect critical motion sequences (\cite{self-calibration-unknown}).

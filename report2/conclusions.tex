% vim:ft=tex
% rubber: module xelatex

\section{Conclusions}
\label{sec:conclusions}
We have implemented image rectification based on the our earlier work
in camera calibration, using the intrinsic and extrinsic parameters from the
calibration to compute the rectification. We have also implemented a
dynamic programming based stereo matching algorithm for creating depth maps from
the rectified images.

Taken in tandem, these formed the foundation for our
implementation of PCA-based holistic face
recognition. PCA uses up to six channels of data from the images (RGB, hue,
saturation and depth data), and works with considerable accuracy. We were pleasantly
suprised by the success of PCA on even very small face images, such as $32\times49$ pixels. However, our testing
was limited by the rather homogeneous face database we used (all young males in the
same lighting and image positions).

We have tested our implementation of the different parts and provide accuracy
test results for the different parts. As part of our testing efforts, we have
also improved on the testing of our calibration implementation, and provided
additional results at the beginning of this report.

Overall, our test results show that we achieve reasonable results in each part of our
implementation, but there is room for improvement. We need to find a way to create depth maps
\emph{in reasonable time} on images that have been rectified but \emph{not} scaled and cropped
by preprocessing, as this appears to reduce their quality. Our stereo matching and rectification could in general be improved in quality, and the efficiency and speed of our face recognition algorithms could be increased.

\begin{center}
\line(1,0){300}

\end{center}

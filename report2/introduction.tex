% vim:ft=tex
% rubber: module xelatex

\section{Introduction}
This report documents the second and final part of our efforts to create a computer vision
system capable of performing face recognition using images taken by a 3D camera. In our
first report, we documented the work we have done on image segmentation,
feature extraction, camera calibration and distortion removal. This part
documents our accomplishments in image rectification, stereo matching and face
recognition. Each stage of the project has built on former stages to some degree.

This report is laid out as follows. In section~\ref{sec:improvements}, we make
several refinements and addendums to our last project for this course. In
section~\ref{sec:prior}, we discuss relevant prior work in the field. In
section~\ref{sec:prog}, we discuss our program, implementation and experimental
results. This is divided into discussions of rectification in
subsection~\ref{sec:rectification}, stereo matching in
subsection~\ref{sec:stereo}, and face recognition in
subsection~\ref{sec:face-rec}. We present our findings in those subsections, and
then briefly summarise our conclusions in section~\ref{sec:conclusions}.

Contributors to this work are Toke Høiland-Jørgensen, whose focus has been
rectification and preprocessing of images for the face recognition and
Ben Meadows, whose focus has been stereo matching and PCA.

\begin{center}
\line(1,0){300}
\end{center}

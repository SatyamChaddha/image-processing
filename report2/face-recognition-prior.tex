% vim:ft=tex
% rubber: module xelatex
\subsection{Face recognition}
\label{sec:face-rec-prior}

\emph{Face recognition} is a difficult problem for the field of image
processing. Faces are themselves extremely variable, and images of them can
differ extremely in positioning, lighting, noise, colour, expression,
adornment\footnote{Spectacles, facial hair, piercings, makeup, and so on.} and other
variables.

Face recognition has typically used either the extraction of face features and
the geometric relationships between them, or `holistic' recognition, using the
whole image for matching, \cite{eigenfaces} although hybrid approaches have also
been proposed - for example, by \citet{zhao03:_face}.

It is possible to create models that represent faces as points in
high-dimensional spaces; similarities between faces means that a reduction of
dimensionality is possible. The most prominent method for determining a
low-dimensional subspace modelling faces in a high-dimensional image space is
Principal Component Analysis (PCA). \cite{eigenfaces}

PCA uses inherent linear transformations to extract principal directions (basis
vectors) of the image subspace. \cite{pca} The basis vectors are determined by
finding the directions of maximum variance in the face images. PCA minimises the
error between the original image and the one reconstructed from its
low-dimensional representation.

The optimal low-dimensional space is that defined by the ``best'' eigenvectors
(those associated with the highest eigenvalues) of the covariance matrix of the
data. These are the \emph{principal components}. \cite{pca}

Various refinements of PCA have been proposed. For example:
\begin{itemize}
\item ICA, independent component
analysis (a generalisation of PCA), \cite{bartlett02:_face}
\item Kernel PCA (a non-linear extension of PCA), \cite{kim02:_face}
\item 2D PCA (PCA using a 2D matrix rather than a 1D vector of values). \cite{yang04:_two_pca}
\end{itemize}
All of these refinements show varying degrees of improvement over straightforward PCA, but with
varying tradeoffs in either computational costs or robustness towards changing
conditions. \cite{yang04:_two_pca}

In spite of the many refinements and different approaches, effective face
recognition, especially for images taken in an outdoor environment and/or under
differing conditions (especially different lighting) remains an open problem.
\cite{zhao03:_face}

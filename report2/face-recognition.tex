% vim:ft=tex
% rubber: module xelatex

\subsection{Face recognition}
\label{sec:face-rec}
We have implemented face recognition based on PCA analysis of the image pixel
values (the holistic approach). Face recognition is effectively the culmination of all our work so far.

Our first step is to preprocess the image database. Images must first be rectified (see section~\ref{sec:rectification}). Preprocessing ensures that a fixed triple of face features (the outer corners of the eyes, and the point between the base of the nose and the lip) are in the same positions for all images. For further details, see section~\ref{sec:preprocess} below. PCA is performed on these processed images.

Up to six channels of data from each image can be used: values for red, green and blue, values for hue and saturation, and depth values. Depth values are created by running our stereo matching algorithm (see section~\ref{sec:stereo}) on the preprocessed images to create equivalent-sized depth maps.\footnote{ It is important at this step to use a fixed multiplier for disparities in all the images, rather than dynamically calculating how to best map the disparities in an image to [0...255]. This is so that representations of relative distances are fixed across all images in the dataset.}

For each pixel in the image in turn, the $n$ data points are concatenated into a series of values, and appended to the single column image vector. The result is a single vector of length $L = n \times m$, where $m$ is the total number of pixels in the image.

For the training phase of the PCA processing, the average image vector for the
training set is computed, and subtracted from each image vector to create the normalised vectors. Each vector is multiplied by its inverse, and the resulting matrices are summed to acquire the overall covariance matrix. This matrix essentially holds all the information of the database of images.

The next step requires that the eigenvectors and eigenvalues of the matrix be calculated. Because the size of the matrix is $L^{2}$, this is computationally expensive. We were able to achieve reduction of dimensionality by using the OpenCV PCA library functions in our code. OpenCV implements the variation introduced by \citet{turk_pentland}, whereby the eigenvectors and eigenvalues are calculated by a square matrix whose dimensionality is equal to the the number of images in the database - a considerable reduction. This allows us to achieve training and matching in a reasonable time and with reasonable memory consumption.

The highest eigenvalues represent those components (`directions') for which we have the greatest variance, i.e. discriminating power. The eigenvectors associated with those highest eigenvalues are kept. The number to keep is a user-configurable parameter. The program calculates the fractional quantity $Q$ of face information being stored by dividing the sum of the \emph{kept} eigenvalues by the sum of \emph{all} the eigenvalues. The user can therefore trade off efficiency (in terms of memory and time) against quality by choosing to keep a number of components such that $Q \geq t$ for some threshold $t$.

The result is a set of eigenvectors which give us an orthonormal basis, or higher-dimensional `principal component space'. Each training image is projected into this space, and for each class of images (i.e. each person whose image was taken), the \emph{average image vector} (in the principal component space) is computed. The maximum distance between
the average image and the set of original training images for that class is also calculated. This maximum distance
(multiplied by some user-configurable factor) is used as a cutoff distance when
matching; images that do not match any classes within the cutoff distance are
considered a no-match.

The program uses this same metric to gauge the confidence in a prediction. Images which, when projected into the orthonormal basis, are close to the edge of the `point cloud' of training examples for the closest-fitting class, are less likely to have been \emph{correctly} matched to that class than those very near to the average image vector.

To test the accuracy of the classification, each training image is projected
into the eigenspace and back; the error between the original image and the
reconstructed image (as the square of the distance) is calculated. Sample results for
this are shown in section~\ref{sec:pcaresults} below.

When given an image to classify, matching is performed as follows. The image vector is constructed as for the training set, using $n$ channels and resulting in a vector length $n \times image_height \times image_width$. The overall average vector is subtracted, and the vector is projected into the PCA space. The distance of this projection to each class average vector is computed. The class vector that is nearest to the test image is considered a match if it is closer than the cutoff depth for that class. Our program by default returns the closest match and the next-closest match, stating the confidence in each guess.

\subsubsection{Preprocessing of images}
\label{sec:preprocess}

Before doing PCA analysis on the face images, they are preprocessed or `registered' to make sure
face features are in (almost) the same positions on all the images. Initially, we operate on rectified images, i.e. stereo images that have
been projected into a geometry in which they have corresponding horizontal epipolar lines. The
rectification step helps diminish unwanted deformation caused by the later steps of the
preprocessing, and also prepares the images for stereo mapping if we wish to use depth data for PCA.

After rectification, face features are manually identified. The features
identified are the outer corners of the eyes, and the point above the lip on the vertical centre of the nose 
(since these features are fairly definable, and seem to be in approximately the same position regardless
of facial expression). For each of these feature points $p$, the average $p'$of each feature point over the whole image
training database is found. Each image is geometrically transformed so that its three feature points align to the location of $p'$.

This is done by solving (for each image) the affine transformation needed to transform the feature point
locations for that image into the average location $p'$, using singular value
decomposition of the resulting set of matrices. This is applied as an affine
transformation to the whole image. Pixel interpolation is performed to create the new image. All three steps are done using OpenCV functions.

Following the transformation, the images are cropped to a user-configurable
ratio. The ratio is determined by adding some factor of the vertical and horizontal distances
between the feature points to the edges of the image. An ellipse is then fitted to the boundaries of the cropped image. Everything outside this ellipse is
`blacked out' to zero values. Finally, the image is scaled down to a fixed width (also
user-configurable), to speed up the processing.

Finally, we apply our stereo matching implementation to the processed images to
yield the depth map that can optionally be used in the PCA matching. The last channels - Hue and Saturation values - are extracted from each pixel by a function during the processing.

\subsubsection{Testing face recognition}
\label{sec:pcaresults}

Methodology: for each test, use whatever number of components means keeping at least 80\% image information (in terms of fraction of the total eigenvalues). Set error threshold to a relatively strict $1.0$, i.e. only identify test images as belonging to a class if the variation between them and the class mean is less than or equal to the maximum variation within the training images for that class.

Consider Table~\ref{tbl:face-rec-1}. The amount of image information being retained when a certain number of components are kept is not heavily dependent on the image size. There is little difference between sizes. In all cases, 12 or 13 components are required for the 80\% information threshold we use in our other experiments.

\begin{table}[h!]
  \centering
  \begin{tabular}{c c c c c}
    \toprule
    \textbf{ } & \textbf{} & \textbf{Image size} & \textbf{} & \textbf{}\\
    \textbf{ } & \textbf{ 32x49 } & \textbf{ 64x99 } & \textbf{128x198} & \textbf{256x396}\\
    \midrule
    \textbf{2 components} & 39.6\% & 39.9\% & 38.8\% & 37.8\% \\
    \textbf{4 components} & 55.6\% & 56.0\% & 55.5\% & 53.2\% \\
    \textbf{6 components} & 64.8\% & 65.2\% & 64.0\% & 62.2\% \\
    \textbf{8 components} & 71.9\% & 72.4\% & 70.5\% & 69.3\% \\
    \textbf{10 components} & 77.2\% & 77.6\% & 76.1\% & 74.6\% \\
    \textbf{12 components} & 80.8\% & 81.3\% & 79.8\% & 78.4\% \\
    \textbf{14 components} & 83.8\% & 84.4\% & 82.3\% & 81.6\% \\
    \textbf{16 components} & 86.3\% & 86.8\% & 85.7\% & 84.2\% \\
    \bottomrule
  \end{tabular}
  \caption[Information retained in face recognition tests for different-sized images]{Information retained in face recognition tests for different-sized images. Each row is a case where a different number of components were kept. Each column is a different image dimensionality. Results are presented as percentage of total summed eigenvalues. In each case, PCA was performed on all six possible channels, on a training set of 44 images.}
  \label{tbl:face-rec-1}
\end{table}

Consider Table~\ref{tbl:face-rec-2}. As expected, we have better reconstruction when we keep more principal components. The total improvement when we go from 4 components to 10 is 41695. The total improvement when we go from 10 components to 16 is 12489. This apparent diminishing returns is because we start including more `important' (in terms of their associated eigenvalue) eigenvectors, and then start adding ones which lend less to the PCA.

\begin{table}[h!]
  \centering
  \begin{tabular}{c c c c}
    \toprule
    \textbf{ } & \textbf{4 components} & \textbf{10 components} & \textbf{16 components} \\
    \textbf{ } & \textbf{(53.2\% image data retained)} & \textbf{(74.6\% image data retained)} & \textbf{(84.2\% image data retained)} \\
    \midrule
    \textbf{Class A} & 9848 & 9018 & 6670\\
    \textbf{Class B} & 11238 & 7853 & 6170\\ 
    \textbf{Class C} & 11719 & 7811 & 7376\\ 
    \textbf{Class D} & 12061 & 7736 & 6560\\
    \textbf{Class E} & 10268 & 7873 & 6383\\
    \textbf{Class F} & 13339 & 8683 & 7001\\ 
    \textbf{Class G} & 10252 & 9022 & 6516\\ 
    \textbf{Class H} & 14167 & 7331 & 5709\\
    \textbf{Class I} & 11945 & 10223 & 7379\\
    \textbf{Class J} & 11196 & 9187 & 6075\\ 
    \textbf{Class K} & 11586 & 8267 & 7596\\
    \bottomrule
  \end{tabular}
  \caption[Sample reconstruction error for the 256x396 size images]{Sample reconstruction error for the 256x396 size images, with PCA performed on all six possible channels. A training set of 44 images was used. Each column is the error for one class (person) in the training set. Error is measured in absolute difference between the original training example image vector and the reconstructed vector projected back out of the PCA space. Each row is a case where a different number of components were kept, with the quantity of information retained (as a percentage of total summed eigenvalues) in brackets.}
  \label{tbl:face-rec-2}
\end{table}

Consider Table~\ref{tbl:face-rec-3}. Note that most often, PCA failed on the classes or test images for the person whose hand was over his face, the person who had the most different positions and expressions, the person who had only two training images, or the person wearing a hat. These are classes F, C, H, K respectively in Table~\ref{tbl:face-rec-2}. Very few test images outside of these unexpected cases were misclassified or unclassified. Note that classes F and H have the highest reconstruction error when only a few components are kept: PCA is unable to reconstruct the considerable variance \emph{within} the class at a lower dimensionality.

\begin{table}[h!]
  \centering
  \begin{tabular}{c c c c c}
    \toprule
    \textbf{ } & \textbf{} & \textbf{Image size} & \textbf{} & \textbf{}\\
    \textbf{ } & \textbf{ 32x49 } & \textbf{ 64x99 } & \textbf{128x198} & \textbf{256x396}\\
    \midrule
    \textbf{Number} & {} & {} & {} & {} \\
    \textbf{successfully} & 30 & 32 & 33 & 34\\
    \textbf{classified} & {} & {} & {} & {} \\
    \midrule
    \textbf{Number} & {} & {} & {} & {} \\
    \textbf{failed to} & 14 & 9 & 7 & 4\\
    \textbf{classify} & {} & {} & {} & {} \\
    \midrule
    \textbf{Number} & {} & {} & {} & {} \\
    \textbf{partially} & 0 & 2 & 3 & 4\\
    \textbf{misclassified} & {} & {} & {} & {} \\
    \midrule
    \textbf{Number} & {} & {} & {} & {} \\
    \textbf{completely} & 0 & 1 & 1 & 2\\
    \textbf{misclassified} & {} & {} & {} & {} \\
    \bottomrule
  \end{tabular}
  \caption[Results of classification testing for different-sized images]{Results of using PCA for face recognition of new images (classification testing). Four different image sizes are tested. A training set of 44 images was used, and in all cases, PCA was performed on all six possible channels. Each column is ... Each row is...}
  \label{tbl:face-rec-3}
\end{table}

Note in Table~\ref{tbl:face-rec-3} that as we increase image size, we see better classification, and also transference of complete failures to classify into misclassifications. By the biggest size, more images are misclassified than unable to classify.



